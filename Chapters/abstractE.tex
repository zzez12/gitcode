\begin{englishabstract}
Segmentation is a basic foundation of computer graphics, and excellent segmentation results will benefit many specific applications in computer graphics and its related areas. Although there are numerous automatic or semi-automatic segmentation algorithms, the algorithms that can be applied to any case do not exist. Some prior knowledge or information of the model can be extracted in advance based on different applications and specific models. Taking advantage of this information will contribute to solve specific segmentation problems greatly. Based on the observation, the article conducted a detailed-analysis of the characteristics of the applications and explore some prior knowledge in advance. By using such knowledge, appropriate and reasonable segmentation algorithms were proposed, and the related applications were better completed.

First, we address the two-dimensional over-segmentation problem using sampling method. Noticed that most of the traditional sampling algorithms are point-based, and they have the characteristics of isotropic. We consider a rarely studied sampling problem--the sampling problem based on object similar to elongated models. We mainly focus on the sampling problem based on lines or segments. Extending the centroidal voronoi tessellation(CVT) into segment-CVT(CVT based on segments), non-isotropic segment sampling results can be generated in 2D space. And we also design an algorithm to generate multi-class sampling results. These algorithms are based on energy optimization frameworks, and have a reasonable running time for all the testing data. Since the sampling problem can be seen as a kind of distribution in space, it can easily generate over-segmentation results. Experimental results show that our sampling algorithm can generate near uniform segment sampling result of quasi blue noise characteristics. Using the definition of Voronoi diagram, the corresponding over-segmentation can be directly produced. Meanwhile, the results can benefit the applications of object placement and texture generation.

The second aspect of our contributions is to propose a new method for details extraction of three-dimensional relief surfaces. We observed that on the relief surface, the details and the base layer have different geometry properties when smoothing the original relief mesh. Based on the observation, we smoothed the surface and then find the initial base points that have less changes during the smooth operation. Then we iteratively do the process as follows, interpolate the base mesh using the obtained base points, and redetermine the base points by the changes of position and normal variation. After several iterations, our algorithm can provide more accurate base points, and the detail part, i.e., the points that are not on the base mesh can be extracted. Experimental results show that our method extract the detail part of relief correctly for the base point dominate relief surfaces.

Finally, a special application of segmentation in video processing was studied, namely, we proposed a new system for stop-motion animation production. Comparing to the traditional frame-by-frame production methods, our system of stop-motion animation is cheaper, easier to build up, more flexible and more intuitive for amateur users, and the interface of our system is capable of animating vast majority of everyday object. Technically, we use a two-phase keyframe-based capturing procedure to get the data. In the first capturing, users need to snatch the model to move the model freely, and in the second capturing process, only some keyframes need to be recaptured by a different occlusion. Using the captured two-phase frames, we segment the occlusion parts and complete them for the keyframes. And using the coherence and smoothness of video, the non-keyframes can be completed by the completed keyframes. We have shown that, our system is novel, effective and efficient, even for amateur animators they can generate high quality stop-motion animations of a wide variety of objects.
%In order to make amateur animators to be able to produce their own stop-motion animations, the system used everyday objects and home-video camera to produce them. Different with the traditional means of stop-motion generating methods that they need capture the scene one frame by one frame, the users of our system can

\englishkeywords{prior knowledge, image segmentation, segment-CVT, quasi blue noise, relief extraction, base surface estimation, stop-motion animation, image completion}

\end{englishabstract}
